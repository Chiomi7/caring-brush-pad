%! TEX root = ../main.tex
\documentclass[main]{subfiles}

\begin{document}

\section{色の違い}

 縁と中心の両方において,RとGの値には大きな違いはなかったが,Bの値に大きな違いがみられた.
使用期間が長くなるにつて,Bの値が大きくなり,白色に近づいた.
 また,HSVを比較した場合,H(色相),V(明度)に大きな違いはなかった.
しかし,S(彩度)は使用期間が長くなるほど,値が小さくなり灰色(無彩色)に近づいた.


\begin{table}[h]
    \caption{RGBの値(縁)}
    \label{table:RGB1}
    \centering
\begin{tabular}{c|c|c|c}
    & R値 & G値 & B値 \\ \hline
   未 低倍率 & 225 & 210 & 75 \\ \hline
   高倍率 & 232 & 212 & 58 \\ \hline\hline
   10A 低倍率 & 222 & 209 & 88 \\ \hline
   高倍率 & 229 & 213 & 87 \\ \hline
   10B 低倍率 & 223 & 214 & 108 \\ \hline
   高倍率 & 214 & 210 & 136 \\ \hline\hline
   長A 低倍率 & 213 & 212 & 150 \\ \hline
   高倍率 & 209 & 210 & 160 \\ \hline
   長B 低倍率 & 211 & 211 & 151 \\ \hline
   高倍率 & 207 & 213 & 178 \\ \hline\hline
\end{tabular}    
\end{table}

\begin{table}[h]
    \caption{RGBの値(中心)}
    \label{table:RGB2}
    \centering
\begin{tabular}{c|c|c|c}
    & R値 & G値 & B値 \\ \hline
   10 低倍率 & 228 & 211 & 74 \\ \hline
   高倍率 & 231 & 212 & 73 \\ \hline\hline
   長 低倍率 & 228 & 214 & 92 \\ \hline
   高倍率 & 230 & 215 & 92 \\ 
\end{tabular}    
\end{table}

\begin{table}[h]
    \caption{HSVの値(縁)}
    \label{table:HSV1}
    \centering
\begin{tabular}{c|c|c|c}
    & H値 & S値 & V値 \\ \hline
   未 低倍率 & 26 & 197 & 168 \\ \hline
   高倍率 & 26 & 192 & 232 \\ \hline\hline
   10A 低倍率 & 27 & 155 & 222 \\ \hline
   高倍率 & 26 & 157 & 229 \\ \hline
   10B 低倍率 & 27 & 133 & 224 \\ \hline
   高倍率 & 29 & 93 & 215 \\ \hline\hline
   長A 低倍率 & 32 & 78 & 215 \\ \hline
   高倍率 & 33 & 64 & 211 \\ \hline
   長B 低倍率 & 35 & 77 & 214 \\ \hline
   高倍率 & 39 & 43 & 213 \\
\end{tabular}    
\end{table}

\begin{table}[h]
    \caption{HSVの値(中心)}
    \label{table:RGB2}
    \centering
\begin{tabular}{c|c|c|c}
    & H値 & S値 & V値 \\ \hline
   10 低倍率 & 27 & 174 & 228 \\ \hline
   高倍率 & 36 & 175 & 231 \\ \hline\hline
   長 低倍率 & 27 & 163 & 229 \\ \hline
   高倍率 & 26 & 153 & 231 \\ 
\end{tabular}    
\end{table}

\end{document}