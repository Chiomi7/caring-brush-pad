%! TEX root = ../main.tex
\documentclass[main]{subfiles}

\begin{document}

% \begin{listing}
\chapter*{付録}

\definecolor{LightGray}{gray}{0.9}

\begin{figure}
    
\begin{minted}
[
frame=lines,
framesep=2mm,
baselinestretch=1.2,
bgcolor=LightGray,
fontsize=\footnotesize,
linenos
]{py}
import cv2

# 対象画像読み込み
img = cv2.imread("img.jpg",cv2.IMREAD_COLOR)

# RGB平均値を出力
# flattenで一次元化しmeanで平均を取得 
b = img.T[0].flatten().mean()
g = img.T[1].flatten().mean()
r = img.T[2].flatten().mean()

# RGB平均値を取得
print("B: %.2f" % (b))
print("G: %.2f" % (g))
print("R: %.2f" % (r))

# BGRからHSVに変換
imgHsv = cv2.cvtColor(img,cv2.COLOR_BGR2HSV)

# HSV平均値を取得
# flattenで一次元化しmeanで平均を取得 
h = imgHsv.T[0].flatten().mean()
s = imgHsv.T[1].flatten().mean()
v = imgHsv.T[2].flatten().mean()

# HSV平均値を出力
# uHeは[0,179], Saturationは[0,255],Valueは[0,255]
print("Hue: %.2f" % (h))
print("Salute: %.2f" % (s))
print("Value: %.2f" % (v))

\end{minted}
\label{fig:label}
\caption*{ソースコード}
\end{figure}

