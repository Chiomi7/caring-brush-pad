%! TEX root = ../main.tex
\documentclass[main]{subfiles}

\begin{document}

\chapter{結論}
カーリングブラシパッドが使用期間によってどれほど摩耗するかを重さ,表面繊維,色の違いから調査した.

肉眼による観察ではブラシパッドにゴミがどれほど付着しているかしか判断できなく,ブラシパッドの表面に摩擦による変化
がまだ見られなかった.しかし,デジタルマイクロスコープを用いて観察すると使用期間が長くなるにつれて表面の繊維は
摩耗し,長期間使用したものにはフィブリル化が見られた.使用期間が長いものほどゴミの付着量は多くみられたが,電子天秤
を用いて重さを測定したところ実際には長期間使用したものほど質量は小さくなっていた.使用するごとに繊維が摩耗して
質量が小さくなってしまっていることがわかった.

表面繊維の観察から使用期間が長くなるほど表面が粗くなっていることが
わかった.未使用のものは繊維が硬く制約されているので繊維の自由度が小さく,他の物体との接触時に摩擦が発生しやすく
なる.しかし長期使用したものは表面繊維が摩耗してより柔軟になっており,自由度が大きくなっていることが考えられる.
そのため,同じ条件でそれぞれのサンプルに摩擦力を調べた際に摩擦係数に違いがみられると考えられる.より精密な条件
の下で実験を行った場合,どれほどの使用期間でブラシパッドの性能に差が出るかを判断することができると言える.




\end{document}