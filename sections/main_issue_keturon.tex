%! TEX root = ../main.tex
\documentclass[main]{subfiles}

\begin{document}

\chapter{結論}
カーリングブラシパッドが使用回数によってどれほど摩耗するかを重さ,表面繊維,色の違いから調査した.

肉眼による観察では主観的なブラシパッドの汚れの多寡の判断しかできず,摩擦による変化がまだ見られなかった.
しかし,デジタルマイクロスコープを用いて観察すると,使用回数が多くなるにつれて表面の繊維が
摩耗しフィブリル化が見られた.

また,使用回数が多いものほどゴミの付着量は多く見られた.
しかし,電子天秤を用いて重さを測定したところ,長期間使用したものほど質量は小さかった.
これらのことから,カーリングブラシパッドを使用するごとに繊維が摩耗して質量が小さくなっていることが分かる.

表面繊維の観察から使用回数が多くなるほど表面が粗くなっていることが分かった.
しかし,本実験では各カーリングブラシパッドを用意する際に,完全に同一の条件で準備されたわけではない.
そのため,より正確に測定するには,同一の条件でカーリングブラシパッドを用意し,対照実験を行う必要がある.
対照実験を行うことにより,それぞれの使用回数におけるカーリングブラシパッドの表面性状を,より的確に測定できる.
また,対照実験を行うことにより,各カーリングブラシパッドの摩擦係数に違いが明らかとなると考える.
この摩擦係数の違いは,カーリングブラシパッドの性能に影響を与えると考える.

\end{document}